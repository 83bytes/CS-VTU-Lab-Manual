\chapter{Alarm signal handler}

Write a C/C++ program to set up a real-time clock interval timer using the alarm API.

\section{Theory}

Every signal has a name. These names all begin with the three characters SIG. For example, SIGABRT is the abort signal that is generated when a process calls the abort function. SIGALRM is the alarm signal that is generated when the timer set by the alarm function goes off.

\section{Code}

\lstinputlisting[style=source-file]{10AlarmSignalHandler/alarm_signal_handler.c}

\section{Output}

Open a terminal. Change directory to the file location in both the terminals.

Compile
\begin{lstlisting}[style=shell-command]
$ g++ 10_alarm_signal_handler.c -o 10_alarm_signal_handler.out
\end{lstlisting}

If there are no errors after compilation execute the program using
\begin{lstlisting}[style=shell-command]
$ ./10_alarm_signal_handler.out
\end{lstlisting}

After a delay of approximately 5 seconds, "Hello!!!" will be printed on the screen.
\begin{lstlisting}[style=shell-output]
Hello!!!
\end{lstlisting}

You can use the \emph{time} command to measure the time taken for the execution.
\begin{lstlisting}[style=shell-command]
$ time ./10_alarm_signal_handler.out
\end{lstlisting}

The \emph{time} command will give you an output similar to this.
\begin{lstlisting}[style=shell-output]
Hello!!!

real    0m5.002s
user    0m0.000s
sys     0m0.000s
\end{lstlisting}
